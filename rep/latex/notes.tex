TODOs and ideas:
  FUTURE IDEAS:
  {
    ->Be able to configure amplifier to record either LFPs (< 1 Hz to 300) or neural spikes (300 Hz > 1 kHz??)
    ref. MIT phd..0004
    
    ->programmable gain (100-1000?)
    
    ->Harvest energy from body?
   }
    
  TODOs:
  {
    -> compare LNA to previous articles 
            -->(p.45 in MIT-thesis) 
	    -->p.4 (=132) in [LNA4]
	    -->p.2 (=3932) in ____lna-use-for-spec-comparisons-of-previous-work.pdf	   
  }
  
  Non-obligatory TODOs
  {
    ->include parameters in the same manner as abbreviations
  }
  
  
  From DesIntKret
  {
    Finally there are some things you should keep in mind before you start 
    For  optimizing  your  transistor  geometries,  you  shou ld  start  with  equal  transistor  gate  length, 
    L = 1? m.  For  strong  transistors  you  need  a  large  width,  W,  and  a  small  length,  L.  You  can start  with  W = 10 L.  Weak  transistors  like  the  colu
    mn  transistor,  which  serves  as  an  active load, can have L > W. Since its purpose is to operate as a resistance, you can start with a large 
    L of 5 ? m and a small W of 2 ? m. The reason why we want to use a transistor as a resistance is that a transistor consumes a lot less area, and 
    we can vary the resistance by varying the bias voltage  over  the  gate.  A  poly  resistor  consumes  a  lot  more  area  and  once  it  has  been  etched 
    onto  the  substrate,  its  resistance  value  is  set.  The  initial  length  of  5 ?m  is  more  than  the suggested maximum length for a transistor. This is 
    because we are using it as a resistance. 

    Start  off  with  equal  size  for  all  switches,  and  remember  that  an  NMOS  is  about  2.5  times 
    stronger  than  a  PMOS  with  the  same  geometry.  So  to save  area,  use  NMOS  wherever  you can. 

    The  buffer  transistor  needs  a  high  transconductance to  produce  large  gains.  This  is  achieved 
    by using a small L and a large W. 
  }
